\documentclass[a4paper,UKenglish]{lipics-v2016}

\newcommand{\comp}{\mathrel{\circ}}
\newcommand{\mhe}[1]{\todo[inline,color=green,caption={}]{#1}}
\newcommand{\abb}[1]{\todo[inline,caption={}]{#1}}
\usepackage{todonotes}

\newcommand{\isProp}{\operatorname{isProp}}
\newcommand{\UU}{\mathcal{U}}
\newcommand{\ttt}{\star}
\newcommand{\Empty}{\mathbf{0}}
\newcommand{\unit}{\mathbf{1}}
\newcommand{\bool}{\mathbf{2}}
\newcommand{\inl}{{\operatorname{inl}}}
\newcommand{\inr}{{\operatorname{inr}}}
\newcommand{\idfunc}[1][]{\operatorname{id}_{#1}}
\newcommand{\flip}{\operatorname{flip}}

% for section-numbered lemmas etc., use "numberwithinsect"

\usepackage{microtype}%if unwanted, comment out or use option "draft"

%\graphicspath{{./graphics/}}%helpful if your graphic files are in another directory

\bibliographystyle{plainurl}% the recommended bibstyle

% Author macros::begin %%%%%%%%%%%%%%%%%%%%%%%%%%%%%%%%%%%%%%%%%%%%%%%%
\title{Parametricity and excluded middle}

%% Please provide for each author the \author and \affil macro, even when authors have the same affiliation, i.e. for each author there needs to be the  \author and \affil macros
% \author[1]{Auke B. Booij}
% \author[2]{Mart{\'\i}n H. Escard{\'o}}
% \affil[1]{School of Computer Science, University of Birmingham,
%   Birmingham B15 2TT, UK\\
%   \texttt{abb538@cs.bham.ac.uk}}
% \affil[2]{School of Computer Science, University of Birmingham,
%   Birmingham B15 2TT, UK\\
%   \texttt{m.escardo@cs.bham.ac.uk}}
% \authorrunning{A.\,B. Booij and M.\,H. Escard{\'o}}

% \Copyright{Auke B. Booij and Mart{\'\i}n H. Escard{\'o}}

\author{~}
\authorrunning{~}
\Copyright{~}


\subjclass{F.4.1 Mathematical Logic}% mandatory: Please choose ACM 1998 classifications from http://www.acm.org/about/class/ccs98-html . E.g., cite as "F.1.1 Models of Computation".
\keywords{relational parametricity, dependent type theory, excluded
  middle, constructive mathematics}% mandatory: Please provide 1-5 keywords
% Author macros::end %%%%%%%%%%%%%%%%%%%%%%%%%%%%%%%%%%%%%%%%%%%%%%%%%

%Editor-only macros:: begin (do not touch as author)%%%%%%%%%%%%%%%%%%%%%%%%%%%%%%%%%%
\EventEditors{}
\EventNoEds{2}
\EventLongTitle{}
\EventShortTitle{}
\EventAcronym{}
\EventYear{}
\EventDate{}
\EventLocation{}
\EventLogo{}
\SeriesVolume{}
\ArticleNo{}
% Editor-only macros::end %%%%%%%%%%%%%%%%%%%%%%%%%%%%%%%%%%%%%%%%%%%%%%%

\begin{document}

\maketitle

\begin{abstract}


  It is known that one can prove the existence of non-parametric
  functions by assuming classical axioms, and our work is a converse
  to that.  We prove classical axioms in Martin-Löf Type Theory by
  assuming specific instances of non-parametricity.

\end{abstract}

\section{Parametricity in dependent type theory}

Reynolds' theory of relational
parametricity~\cite{DBLP:conf/ifip/Reynolds83}~characterizes the terms
of the polymorphically typed $\lambda$-calculus System~F.  This theory
of parametricity has since been extended to richer and more expressive
type theories. Bernardy and
Jansson~\cite{DBLP:journals/jfp/BernardyJP12} develop parametricity
for pure type systems, and Atkey et
al.~\cite{DBLP:conf/popl/AtkeyGJ14} develop parametricity for
Martin-L\"of Type Theory (MLTT).  MLTT admits the definition of
polymorphic functions using dependent function types and a
universe~$\UU$.  For example, the polymorphic identity function has
type $\forall \alpha . \alpha \to \alpha$ in System~F, and the type
$\prod_{X : \UU} X \to X$ in MLTT.\@

Reynolds' parametricity results show that the only term of System~F
with type $\forall \alpha . \alpha \to \alpha$ is
$\Lambda \alpha. \lambda (x:\alpha).x$, the identity function at every
type.  Atkey et al.\ prove that any term $f:\prod_{X : \UU} X \to X$
of MLTT satisfies $e \comp f_X =f_Y \comp e$ for arbitrary
$e : X \to Y$, where the equality sign denotes the identity type.
% This is in the style of Wadler~\cite{DBLP:conf/fpca/Wadler89}, who
% calls such conclusions \emph{free theorems}.
From this we can then deduce that
$M$ has to be the identity function at every type in MLTT.

Keller and Lasson~\cite{DBLP:conf/csl/KellerL12} describe a type
theory where parametricity is part of the language, and show the
negation of excluded middle in the type theory.  Also it is easy to
see that if excluded middle is part of the language, then there is a
function $f : \prod_{X : \UU} X \to X$ which is the negation function
$\flip:\bool\to\bool$ at the type $\bool$ of
booleans~\cite[Exercise~6.9]{hottbook}, and so parametricity is
violated.  In this note we show that, conversely, if $f$ is invariant
under isomorphism, as is the case in univalent extensions of MLTT,
this specific violation of parametricity gives excluded middle.

% For arbitrary extensions\todo{models?}~of MLTT it may not be clear
% whether they are parametric.  As an example, in the model of cubical
% sets (a constructive model of type theory with univalence), there may
% be elements in the \emph{interpretation} of the type
% $\prod_{X : \UU} X \to X$ which are not the identity.  Nevertheless,
% as long as such models are constructive, it is difficult to violate
% parametricity.  Specifically, given an element of
% $\prod_{X : \UU} X \to X$ that is not the identity function on the
% booleans, excluded middle follows.  Indeed, this can be proved
% internally in MLTT under the assumption that the polymorphic function
% is invariant under isomorphism of types.  This contrasts with previous
% results that parametricity negates excluded middle.  Here instead we
% get excluded middle by a specific violation of parametricity.

\section{Classical axioms from non-parametricity}

We now list a number of ways in which classical axioms can be derived
from specific violations of parametricity. We work with the version of
excluded middle from univalent foundations, namely that $P + \neg P$
for all propositions $P$, where a type is called a proposition if it
has at most one element, meaning that any two of its elements are
equal.

\subsection{Polymorphic endomaps}

By parametricity, we know that there is no \emph{closed term} of type
$\prod_{X : \UU} X \to X$ other than the identity. This is a meta-theorem \emph{about} type
theory. The following formulates a theorem \emph{of} type theory.  Say
that a function $f:\prod_{X : \UU} X \to X$ is \emph{invariant under
  isomorphism} if for any two types $X$ and $Y$ and any isomorphism
$e:X \to Y$, we have $e \comp f_X = f_Y \comp e$, where we have
written $f_X$ as a shorthand for $f(X)$.
\begin{theorem}
\label{thm:identity-bool}
If there is a function $f:\prod_{X : \UU} X \to X$ such that $f_\bool$
is the flip map and $f$ is invariant under isomorphism, then the law
of excluded middle holds.
\end{theorem}
This holds in any dependent type theory with
an empty type, a unit type~$\unit$, a type of booleans, identity types,
coproducts, $\Pi$-types, and a universe $\UU$.
\begin{proof}
  To derive excluded middle from~$f$, let $P$ be an arbitrary
  proposition. We do case analysis on the value of
  $f_{P+\unit}(\inr(\ttt)) : P+\unit$.
  \begin{enumerate}
  \item If it is of the form $\inl(p)$ with $p:P$, we conclude
    immediately that $P$ holds.
  \item If it is of the form $\inr(\ttt)$, then $P$ cannot hold, for if
    we had $p:P$, then the map $e : \bool \to P + \unit$ defined by $e(0)=\inl(p)$ and $e(1)=\inr(\ttt)$ would be an isomorphism, and hence $e \comp f_\bool = f_{P+\unit} \comp e$ and so
  $ \inl(p) = e(0) = e (f_\bool(1)) = f_{P+\unit} (e(1)) = \inr(\ttt)$, which is a contradiction.
  \end{enumerate}
Therefore $P$ or not $P$.
% % We need to show $P + \neg P$.
%   % The essence of the argument is that we construct a type that is
%   % isomorphic to $\bool$ if $P$ holds, and something else otherwise, in
%   % such a way that we force $f$ to tell us something about $P$.
%   % We pick that type to be $\unit+P$.
%   % Here, $\unit$ is the unit type whose only element is $\ttt:\unit$.
% If $P$ holds, then $P$ is isomorphic to $\unit$,
% and so $\unit+P$ is isomorphic to $\bool$.
% So then, by the fact that $f$ is invariant under isomorphism,
% $f_{\unit+P}$ must flip the elements of $\unit+P$.
% % In other words, since $f_\bool$ does not have any fixed points,
% % neither does $f_{\unit+P}$.
% % Conversely, if $P$ does not hold, then $\unit+P$ is isomorphic to
% % $\unit$, and $f_{\unit+P}$ is the identity map (as it is the only
% % map $\unit+P \to \unit + P$).
%
% So we consider $x:=f_{\unit+P}(\inl(\ttt))$, noting that $x:\unit+P$.
% Using the induction principle of coproducts, we can do case analysis
% on $x$.
% If $x$ is $\inr(p)$ for some $p:P$, then we know that $P$ holds.
% If $x$ is $\inl(t)$ for some $t:\unit$, then, by the induction
% principle of $\unit$, we know that $t=\ttt$, and so $f_{\unit+P}$ has
% a fixed point: so it is impossible that $\unit+P$ is isomorphic to
% $\bool$, because $f_\bool$ has no fixed points.
% From the absurdity of such an isomorphism we may conclude that $P$
% does not hold.
\end{proof}

\begin{remark} \leavevmode
%  Assuming univalent extensions, we can weaken the hypothesis of the
%  theorem to get the same conclusion:
  \begin{enumerate}
  \item The invariance of $f$ under isomorphism follows directly from
    internal parametricity. Hence the theorem gives an alternative
    proof that excluded middle is false in the presence of internal
    parametricity.
  \item If we assume univalence, any such $f$ is automatically
    invariant under isomorphism.  More interestingly, if we assume
    function extensionality, which itself follows from univalence,
    then we can weaken the assumption to assert that $f_{\bool}$ is
    not the identity map. This observation is due to Mike
    Shulman~\cite[comment]{parametricity:and:em} and is seen as
    follows.

    Because $\flip$ is an isomorphism, we must have
    $\flip \comp f_\bool = f_\bool \comp \flip$.  This shows that
    $f_\bool$ cannot be any of the two constant functions
    $\bool \to \bool$.  By the assumption that it is not the identity,
    $f_\bool$ must itself be the flip function, as there are only four functions
    $\bool \to \bool$.
  \end{enumerate}
\end{remark}

We say that a point $x$ of a type $X$ is \emph{isolated} if the type
$x=y$ is decidable.  This construction can be generalized to the
assumption that we are given $f:\prod_{X : \UU} X \to X$ with the
property that there is a type $X:\UU$ with an isolated point $x:X$
with $f_X(x) \neq x$.  For $X$ to have an isolated point $x$ is
equivalent to saying that $X$ is isomorphic to $Y+\unit$ for some type
$Y$ by a map that sends $x$ to $\inr(\ttt)$. The following was
observed independently by Mike
Shulman~\cite[comment]{parametricity:and:em} and ourselves:
\begin{theorem}
  If there is a function $f:\prod_{X : \UU} X \to X$ such that
  $f_X(x)\neq x$ for some isolated point $x:X$ and $f$ is invariant
  under isomorphism, then the law of excluded middle holds.
\end{theorem}

\begin{proof}
  To derive excluded middle from~$f$, let $P$ be an arbitrary
  proposition. We do case analysis on the value of
  $f_{P\times Y + \unit}(\inr(\ttt)) : P \times Y + \unit$.
  \begin{enumerate}
  \item If it is of the form $\inl((p,y))$ with $p:P$, we conclude
    immediately that $P$ holds.
  \item If it is of the form $\inr(\ttt)$, then $P$ cannot hold, for
    if we had $p:P$, then the map $e : X \to P\times Y + \unit$
    defined by $e(x)=\inr(\ttt)$ (where $x$ is the isolated point) and
    $e(y)=\inl((p,y))$ for $y \neq x$ would be an isomorphism, and
    hence $e \comp f_X = f_{P\times Y+\unit} \comp e$ and so
    $ \inl((p,f_X(x))) = e (f_X(x)) = f_{P\times Y+\unit} (e(x)) =
    f_{P\times Y+\unit} (\inr(\ttt)) =\inr(\ttt)$, which is a contradiction.
  \end{enumerate}
Therefore $P$ or not $P$.
%
% We consider $x:=f_{\unit + P\times Y}(\inl(\ttt))$ (where again
% $P$ is our arbitrary proposition).
% This works because if $P$ holds, then $\unit + (P
% \times Y)$ is isomorphic to $X$, and if $P$ does not hold, then
% $\unit + P\times Y$ is isomorphic to $\unit$.
%
% So we do case analysis on $x$.
% If $x$ is $\inr((p,y))$ we know that $P$ holds (as $p:P$).
% If it is $\inl(\ttt)$ we, then we claim that $P$ does not hold.
% This means we assume $P$ holds and derive a contradiction.
% So, assuming that $P$ holds, we can find an isomorphism between $X$
% and $\unit + P\times Y$ that sends $x$ to $\inl(\ttt)$.
% But $f$ could not have been invariant under this isomorphism, as on
% the one hand, $f_X(x)\neq x$, but on the other hand, $f_{\unit + (P
%   \times Y)}(\inl(\ttt))=\inl(\ttt)$: a contradiction.
\end{proof}

In both examples, we have assumed that some given
$f:\prod_{X : \UU} X \to X$ is different from the polymorphic identity
at a specific type $X$.  The assumption that $f$ is not the
polymorphic identity map, without knowing in which way it is
different, does not seem to be strong enough to be able to prove
classical axioms.  It is natural to wonder what the weakest assumption
is that both implies that the given $f$ is not parametric, and can be
used to prove classical axioms, and we leave this as an open question.

\subsection{Church numerals}

The above can be applied to obtain classical axioms from
other kinds of violations of parametricity.  As a simple example,
consider $f:\prod_{X : \UU} (X \to X) \to (X \to X)$.  Parametric
elements of this type are Church numerals.
%
Given $f$, we can define a polymorphic endomap
$g:\prod_{X : \UU} X \to X$ by $g_X=f_X(\idfunc[X])$, where
$\idfunc[X]$ is the identity function.  If $f$ is invariant under
isomorphism, then so is $g$, and hence the assumption that
$f_\bool(\idfunc[\bool])$ is \emph{not} the identity function gives
excluded middle, assuming function extensionality.

\subsection{Maps of the universe into the booleans}

We consider two more violations of parametricity.
A function $f:\prod_{X : \UU} X \to \bool$ is invariant under
isomorphism if we have $f_Y \comp e = f_X$ for any isomorphism
$e:X\to Y$.  Parametricity tells us that $f$ has to be constant.
Consider the violation of parametricity in which there is a type $X$
with points $x,y:X$ with $f_X(x) \neq f_X(y)$.  If we additionally
assume that there is an isomorphism of $X$ which maps $x$ to $y$, we
arrive at the contradiction that $f$ was not invariant under
isomorphism.  So the mere invariance under isomorphism shows that this
specific violation of parametricity is impossible.

A function $f : \UU \to \bool$ is invariant under
isomorphism, or extensional, if whenever $X$ and~$Y$ are isomorphic
types, $f(X)=f(Y)$.  The parametric model of Atkey et al.\ shows that
such functions are constant. Without parametricity assumptions, we
have the following non-constructive conclusion:
\begin{theorem}[Escard{\'o} and Streicher~\cite{DBLP:journals/apal/EscardoS16}]
  If $f$ is invariant under isomorphism, and there are $X,Y:\UU$ with
  $f(X) \neq f(Y)$, then the weak limited principle of omniscience
  holds.
\end{theorem}
Moreover, Alex Simpson strengthened the conclusion to weak excluded
middle, namely $\neg A + \neg\neg A$ for all $A : U$, assuming
function extensionality, as also recorded
in~\cite{DBLP:journals/apal/EscardoS16}.

The results discussed so far suggest that different violations of
parametricity have different proof-theoretic strength: some violations
are impossible, while others imply varying amounts of excluded middle.

\section{Automorphisms of the universe}

There have been attempts to use parametricity to show that the only
automorphism of a universe of types is the identity.  Nicolai Kraus
observed in the HoTT mailing list~\cite{automorphisms:kraus} that,
assuming univalence, automorphisms of a universe $\UU$ living in a
universe $\mathcal{V}$ correspond to elements of the loop space
$\Omega(\mathcal{V},\UU)$.  It can be seen that nontrivial elements of
the higher loop space $\Omega^2(\mathcal{V},\UU)$ imply violations of
parametricity for $\prod_{X : \UU} X \to X$.  This suggests that
parametricity may play a role in automorphisms of the universe.

We are not aware of a proof that parametricity implies that the only
automorphism of the universe is the identity. However, in the spirit
of the above development, we can show that automorphisms with specific
properties imply excluded middle.

As observed by Peter Lumsdaine in the HoTT mailing
list~\cite{automorphisms:lumsdaine:lem}, in the presence of excluded
middle we can prove that there exists a nontrivial automorphism of the
universe.  The map $\UU\to\UU$ is defined as follows.  Given a type
$X$, we use excluded middle to decide if it is a proposition.  If it
is, we map $X$ to $\neg X$, and otherwise we map $X$ to itself.
Assuming propositional extensionality, that is, that any two logically
equivalent propositions are equal, or univalence for propositions,
this map flips the empty type $\Empty$ with the unit type~$\unit$.
% This works because if excluded middle
% holds, then all propositions are either $\Empty$ or $\unit$, and the
% mapping $X \mapsto \neg X$ exchanges these two.
% Conversely, the existence of a nontrivial automorphism can imply
% classical axioms.
We have the following converse:
\begin{theorem}
\label{thm:automorphism}
Assuming propositional extensionality, if there is an automorphism of the
universe that maps the unit type to the empty type, then excluded middle
holds.
%$F:\UU\to\UU$ such that $F(\unit)=\Empty$, then excluded middle holds.
\end{theorem}%\todo{reversed assumption because of Peter's generalization}
% To prove Theorem~\ref{thm:automorphism}, we use the following lemma,
% which is left as an exercise for the reader.

\begin{lemma}
\label{lem:lem-negation}
  Excluded middle holds iff every proposition is logically equivalent
  to the negation of some type.
\end{lemma}
\begin{proof}
  Excluded middle is equivalent to $\neg \neg P \to P$ for all
  propositions $P$, and we always have that
  $\neg\neg\neg A \to \neg A$ for any type $A$.
\end{proof}

Not even an embedding of $\UU$ into itself that maps the unit type to
the empty type is possible without classical axioms:
\begin{lemma} \label{left:cancellable} Assuming propositional
  extensionality, if there is a left-cancellable map $f: \UU \to \UU$
  with $f(\unit)=\Empty$ then excluded middle holds.
\end{lemma}
\begin{proof}
  For an arbitrary proposition $P$, we have:
  \begin{align*}
    P \,\!& \leftrightarrow  P = \unit \quad
    && \mbox{(by propositional univalence)}
    \\
        &\leftrightarrow  f(P)=f(\unit)
    && \mbox{(because $f$ is left-cancellable)}
    \\
        &\leftrightarrow f(P)=\Empty
    && \mbox{(by the assumption that $f(\unit)=\Empty$)}
    \\
        &\leftrightarrow \neg f(P)
          && \mbox{(by propositional extensionality).}
    % P \,\!& \leftrightarrow  P = \unit \quad
    % && \mbox{(by propositional univalence)}
    % \\
    %     &\leftrightarrow  P=F^{-1}(\Empty)
    % && \mbox{(by the assumption that $F(\Empty)=\unit$)}
    % \\
    %     &\leftrightarrow  F(P)=\Empty
    % && \mbox{(because $F$ is left-cancellable)}
    % \\
    %     &\leftrightarrow \neg F(P)
    %       && \mbox{(by propositional univalence)}
  \end{align*}
  Hence $P$ is logically equivalent to the negation of the type
  $f(P)$, and therefore, by Lemma~\ref{lem:lem-negation}, excluded
  middle holds.
\end{proof}
This concludes the proof of Theorem~\ref{thm:automorphism}. Further
assuming univalence, Peter Lumsdaine generalized this as follows.  Say
that a type $A$ is \emph{inhabited} if the unique map $A\to\unit$ is
surjective. This is equivalent to saying that its propositional
truncation $\left\| X \right\|$ is
pointed.  % \item For the isomorphism $F$, we can equivalently require
  %   $F(\Empty)=\unit$ instead of $F(\unit)=\Empty$.  Then we use the
  %   following lemma, where a type $A$ is said to be inhabited if the
  %   unique map $A\to\unit$ is surjective.
\begin{lemma}\label{lem:prop-equivalent}
  If univalence holds and $A$ is an inhabited type, then any
  proposition $P$ is logically equivalent to the identity type
  $P \times A = A$.
\end{lemma}
\begin{proof}
  If $P$ then $P=\unit$ by propositional extensionality, and so
  $P\times A=A$ by univalence.  Conversely, assume $P \times A = A$.
  Then $\|P \times A\| = \unit$ as $A$ is inhabited.  So
  $\|P\|\times\|A\|=\unit$, and hence $P=\unit$.
\end{proof}
Using this, we can weaken the hypothesis to the requirement that $f$
maps some inhabited type to the empty type, and get the same
conclusion as in Lemma~\ref{left:cancellable}, at the expense of
requiring univalence rather than just propositional extensionality:
\begin{lemma} \label{left:cancellable:bis} Assuming univalence, if
  there is a left-cancellable map $f: \UU \to \UU$ with
  $f(A)=\Empty$ for some inhabited type $A$, then excluded middle holds.
\end{lemma}
\begin{proof}
  For an arbitrary proposition $P$, we have:
  \begin{align*}
    P \,\!& \leftrightarrow  P \times A = A \quad
    && \mbox{(by propositional univalence)}
    \\
        &\leftrightarrow  f(P \times A)=f(A)
    && \mbox{(because $f$ is left-cancellable)}
    \\
        &\leftrightarrow f(P \times A)=\Empty
    && \mbox{(by the assumption that $f(A)=\Empty$)}
    \\
        &\leftrightarrow \neg f(P \times A)
          && \mbox{(by propositional univalence).}
    % P \,\!& \leftrightarrow  P = \unit \quad
    % && \mbox{(by propositional univalence)}
    % \\
    %     &\leftrightarrow  P=F^{-1}(\Empty)
    % && \mbox{(by the assumption that $F(\Empty)=\unit$)}
    % \\
    %     &\leftrightarrow  F(P)=\Empty
    % && \mbox{(because $F$ is left-cancellable)}
    % \\
    %     &\leftrightarrow \neg F(P)
    %       && \mbox{(by propositional univalence)}
  \end{align*}
  Hence $P$ is logically equivalent to the negation of the type
  $f(P)$, and therefore, by Lemma~\ref{lem:lem-negation}, excluded
  middle holds.
\end{proof}
\begin{theorem}[Lumsdaine~\cite{automorphisms:lumsdaine:proposition}]
\label{thm:automorphism:bis}
Assuming univalence, if there is an automorphism of the universe that
maps some inhabited type to the empty type, then excluded middle
holds.
%$F:\UU\to\UU$ such that $F(\unit)=\Empty$, then excluded middle holds.
\end{theorem}

\begin{corollary}
  Assuming univalence, if there is an automorphism $g : \UU \to \UU$ of the universe
  with $g(\Empty) \ne \Empty$, then the double
  negation \[ \neg \neg \prod_{P:\UU} \isProp(P) \to P + \neg P \] of
  the principle of excluded middle holds.
\end{corollary}
\begin{proof}
  Let $f$ be the inverse of $g$. If $g(\Empty)$ then
  $\left\| g(\Empty) \right\|$, and because $f$ maps $g(\Empty)$ to
  $\Empty$, we conclude that excluded middle holds by
  Theorem~\ref{thm:automorphism:bis}.  But the assumption
  $g(\Empty) \ne \Empty$ is equivalent to $\neg \neg g(\Empty)$ by
  propositional univalence, and so it implies the double negation of
  excluded middle.
\end{proof}
% In particular, because continuity axioms negate the principle of
% excluded middle, they are inconsistent with the double negation of
% excluded middle. But they are consistent with
% Instead of
% requiring that $F(\Empty)$ is inhabited, we require that
% $F(\Empty)\neq\Empty$.  By propositional univalence, this means
% $\neg\neg F(\Empty)$, which says that $F(\Empty)$ is non-empty.
% Hence, using the previous generalization, we get the weaker conclusion
% of the double negation of excluded middle.

% The double negation of excluded middle is expected\todo{right?}~to be
% unprovable in a univalent type theory without classical axioms.






\subparagraph*{Acknowledgements.}

% The first named author would like to thank Uday Reddy for giving a
% talk on parametricity that inspired this research.  We would like to
% thank Andrej Bauer, Peter Lumsdaine and Mike Shulman for useful
% discussions.


% \abb{Does the first page footer need to be adjusted by us?} No.

\appendix


%%
%% Bibliography
%%

%% Either use bibtex (recommended),

\bibliography{parametricity-lem}

%% .. or use the thebibliography environment explicitely



\end{document}

%%% Local Variables:
%%% mode: latex
%%% TeX-master: t
%%% End:
